\assignementTitle{Исчезновение леса}{10}

Обезлесение является актуальной проблемой во многих частях земного шара, поскольку влияет на 
экологические, климатические и социально-экономические характеристики и снижает качество жизни. 
Довольно часто лес исчезает участками. Благодаря космическим снимкам мы можем определять, 
где лес растет, а где его нет. Каждый снимок с космоса проходит сложную предобработку. 

Для упрощения анализа, мы предлагаем рассмотреть снимки в следующем предобработанном формате: 
пометим каждый пиксель на котором изображен лес символом '.', а тот, на котором нет леса — 'x'.

Ваша задача будет заключаться в том, чтобы посчитать количество участков, в которых лес отсутствует. 
Под участком будем понимать набор таких пикселей без леса, которые касаются минимум в одной точке хотя 
бы с ещё одним пикселем данного набора. Или не касаются ни одного пикселя без леса.

Данную задачу можно было бы решать вручную, однако сколько это займет по времени, если рассматривать 
мелкие-мелкие пиксели на сотнях или даже тысячах изображений? Вы однозначно умеете делать это быстрее!

\inputfmtSection
В первой строке задается целое число $t \space (1 \leq t \leq 100)$  — количество тестов.

Далее с новой строки для каждого теста задается целое число $n\space (1 \leq n \leq 100)$  — размер стороны 
квадратного изображения в пикселях.

И в следующих $n$ строках указывается строка из пикселей ('x' 'и '.') длиной $n$.

\outputfmtSection
В отдельной строке $ t $ целых чисел через пробел — количество горящих участков для каждого из изображений.

\markSection

Баллы за задачу будут начисляться пропорционально количеству успешно пройденных тестов.

Примерно в $40\%$  тестов $50 \leq t \leq 100$.  
Примерно в $75\%$  тестов $50 \leq n \leq 100$. 

\sampleTitle{1}

\begin{myverbbox}[\small]{\vinput}
3
10
..x.......
..........
........x.
....x.....
......x...
..........
...x.....x
....x...x.
xx...x....
.........x
10
xxxxxxxxxx
xxxxxxxxxx
xxxxxxxxxx
xxxxxxxxxx
xxxxxxxxxx
xxxxxxxxxx
xxxxxxxxxx
xxxxxxxxxx
xxxxxxxxxx
xxxxxxxxxx
10
..........
..........
..........
..........
..........
..........
..........
..........
..........
..........
\end{myverbbox}

\begin{myverbbox}[\small]{\voutput}
8 1 0 
\end{myverbbox}
\inputoutputTable

\includeSolutionIfExistsByPath{1st_tour/inf/try_1/task_10}