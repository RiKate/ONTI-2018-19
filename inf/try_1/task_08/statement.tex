\assignementTitle{Полиномиальный характер}{10}

В результате некоторого эксперимента исследователь получил график. Он обратил внимание, что график описывает некоторую непрерывную функцию, которая пересекает ось абсцисс в некотором количестве точек. Ещё он отметил, что нет таких участков, в которых график касается оси абсцисс, не пересекая её. А значения функции на левом конце графика всегда отрицательны.

У исследователя возникло предположение, что данная функция полиномиальная. И для того, чтобы это проверить, он решил записать все значения, в которых функция пересекла ось абсцисс и построить по данным значениям полиномиальную функцию, а дальше наложить графики и оценить расхождение. Помогите исследователю составить полиномиальную функцию, проходящую через заданные точки. При этом порядок данной функции не должен быть выше количества точек, а коэффициент перед старшей степенью должен быть по модулю равен 1.

\inputfmtSection
Программа в первой строке принимает на вход количество точек $n\space(1\leq n\leq 10)$.

Далее в следующей строке вводятся $n$  аргументов $x_i \space (-100 \leq x_i \leq 100; x_i \in Z; \forall i, j:x_i \neq x_j)$, в которых функция принимает нулевое значение. 

\outputfmtSection
В качестве ответа выведите в отдельной строке через пробел $n+1$ целочисленных коэффициентов полинома, начиная со старшей степени.

\markSection

Баллы за задачу будут начисляться пропорционально количеству успешно пройденных тестов.

\sampleTitle{1}

\begin{myverbbox}[\small]{\vinput}
    2
    -5 3
\end{myverbbox}

\begin{myverbbox}[\small]{\voutput}
    -1 -2 15
\end{myverbbox}
\inputoutputTable

\includeSolutionIfExistsByPath{1st_tour/inf/try_1/task_08}