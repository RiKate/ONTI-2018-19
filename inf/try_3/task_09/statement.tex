\assignementTitle{Reverse engineering}{10}

Витя написал программу, которая вычисляет функцию вида $f(x)=(ax^2+bx+c)\space mod\space d$, где $mod$ — остаток от деления. 
Для $25$ различных значений аргументов из диапазона от $0$ до $24$ включительно он узнал, чему равна данная функция. Из этих 
данных он составил тесты к этой задаче, а саму функцию забыл. Автор задачи точно уверен в том, что значения $a$, $b$, $c$ и $d$ — 
неотрицательные целые числа, не превосходящие $10$.

Помогите Вите восстановить эту функцию. Напишите программу, которая проходит все тесты.

В данной задаче нет ограничение на количество посылок. Ваша задача "смайнить" тесты и узнать, что же за функция зашита в программе. 
Как это сделать? Учитесь правильно ошибаться! 

Для того, чтобы вы могли смотреть вердикты по всем тестам, вы обязательно должны проходить верно первый тест.

\inputfmtSection
Целое число $N\space(0\leq N\leq24)$  — аргумент функции.

\outputfmtSection
Единственное целое число — значение функции при заданном аргументе.

\markSection

Баллы за задачу будут начисляться пропорционально количеству успешно пройденных тестов.

\sampleTitle{1}

\begin{myverbbox}[\small]{\vinput}
    6
\end{myverbbox}

\begin{myverbbox}[\small]{\voutput}
    4
\end{myverbbox}
\inputoutputTable

\includeSolutionIfExistsByPath{1st_tour/inf/try_3/task_09}