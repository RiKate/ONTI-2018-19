\assignementTitle{Круглый стол}{20}

Организаторы одного из турниров по одной из самых известных карточных игр заказали круглый стол, 
за которым будет сидеть несколько (минимум 2) игроков во время каждой из игр.

Профсоюз игроков этой карточной игры выдвинул условие: каждый игрок на время игры 
должен иметь личное пространство как минимум длины $d$.

Если посчитать стол кругом, то длина пространства игрока определяется как длина хорды, 
которая отделяет сегмент круга, на котором игрок может положить свои карты и руки, от 
остальной части стола.

Помогите организаторам определить, какое максимальное количество игроков может одновременно 
сидеть за этим столом.

\inputfmtSection

Вводится два целых числа $d$ и $r$ $(1 \leq d, r \leq 2 \cdot 10^{9})$ - минимальная длина личного 
пространства и радиус круглого стола.

\outputfmtSection

Если нельзя усадить за стол даже двух игроков, выведите $-1$, иначе выведите единственное 
положительное число - максимальное количество игроков, которых можно разместить за столом.

\markSection

Баллы за задачу будут начисляться пропорционально количеству успешно пройденных тестов.

\sampleTitle{1}

\begin{myverbbox}[\small]{\vinput}
    6 3
\end{myverbbox}

\begin{myverbbox}[\small]{\voutput}
    2
\end{myverbbox}
\inputoutputTable

\sampleTitle{2}

\begin{myverbbox}[\small]{\vinput}
    7 2
\end{myverbbox}

\begin{myverbbox}[\small]{\voutput}
    -1
\end{myverbbox}
\inputoutputTable

\sampleTitle{3}

\begin{myverbbox}[\small]{\vinput}
    12 1000
\end{myverbbox}

\begin{myverbbox}[\small]{\voutput}
    523
\end{myverbbox}
\inputoutputTable

\sampleTitle{4}

\begin{myverbbox}[\small]{\vinput}
    3 3
\end{myverbbox}

\begin{myverbbox}[\small]{\voutput}
    6
\end{myverbbox}
\inputoutputTable

\includeSolutionIfExistsByPath{1st_tour/inf2/try_3/task_02}